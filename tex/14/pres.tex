%
% -- Manlio Modugno

\documentclass{beamer} 
\usepackage{eulervm}
%\usepackage{booktabs}
\usepackage{listings}
\usepackage{bold-extra}
\usepackage{cancel}
\usepackage{fancybox}
\usepackage{soul}
\usepackage[english]{babel}
\usepackage[utf8]{inputenc}
\usepackage{hyperref}
\usepackage{amsmath}
%\hypersetup{colorlinks=true,urlcolor=blue}

\newcommand{\codefont}{\fontsize{6}{8}\selectfont}
\lstset{language=[Sharp]C, 
captionpos=b, 
frame=lines,
lineskip= 1pt, %space between lines
basicstyle=\codefont, 
keywordstyle=\color{blue}, 
commentstyle=\color{green}, 
stringstyle=\color{red}, 
numbers=left, 
numberstyle=\tiny, 
stepnumber=2,
numbersep=5pt,
breaklines=true, 
breakatwhitespace=false,
showstringspaces=false,
frame=single,
tabsize=2,
emph={double,bool,int,unsigned,char,true,false,void},
emphstyle=\color{blue},
emph={Assert,Test},
emphstyle=\color{red},
emph={[2]\using,\#define,\#ifdef,\#endif},
emphstyle={[2]\color{blue}}
}


\mode<presentation>
\definecolor{title_color}{RGB}{2,128,181} 
\usetheme{Ilmenau}
\usecolortheme[named=title_color]{structure}
\setbeamercolor{palette quaternary}{use=structure,fg=black,bg=white} %header footer color
\useoutertheme[subsection=false]{smoothbars}
\setbeamercovered{transparent}
\setbeamertemplate{navigation symbols}{}
\setbeamerfont{subsection in toc}{size=\scriptsize}

\title{Bad Smells in Code / Refactorings - part3}
\author{Manlio Modugno}
\institute[GMTechnologies] 

\date[01.12.2016] 
{01.12.2016 - Bad Smells in Code / Refactorings - part3}

\subject{}

\graphicspath{{img/}}
\pgfdeclareimage[height=0.6cm]{mfg-logo}{img/mfgLogo}
\logo{\pgfuseimage{mfg-logo}}

%
% Content start
%
\begin{document}
\begin{frame}
  \titlepage
\end{frame}

\begin{frame}
  \frametitle{Argomenti Trattati}
  \tableofcontents
\end{frame}

\section{Incomplete Library Class}
\subsection{Incomplete Library Class}
\begin{frame}
  \frametitle{Incomplete Library Class}
  \begin{itemize}
	\item<+-> Often we use library (or framework..) to accomplish our job..
	\item<+-> But they are omniscient and they can't save the world.. 
	\item<+-> ..it's really difficult figure out a design until we've mostly built it
	\item<+-> tie to an external dependency can be dangerous.. you can't acess the code
	\item<+-> To avoid these kind of situations use \textit{Introduce Foreign Method} and \textit{Introduce Local Extension}
   \end{itemize}
\end{frame}

\subsection{Introduce Foreign Method}
\begin{frame}
  \frametitle{Introduce Foreign Method}
  \begin{itemize}
	\item<+-> When modify code of external library is impossible, wrap the behaviour you need in the client class..
	\item<+-> This is always a workaround.. so you should signal to the code owner to add that behaviour..
   \end{itemize}
\end{frame}

\subsection{Introduce Local Extension}
\begin{frame}
  \frametitle{Introduce Local Extension}
  \begin{itemize}
	\item<+-> If you ara forced to introduce many foreign methods or you need often the same foreign method...
	\item<+-> ..consider to introduce a class that wraps (NO extends !) the dependency introducing a local extension
   \end{itemize}
\end{frame}

\section{Data Class}
\subsection{Data Class}
\begin{frame}
  \frametitle{Data Class}
  \begin{itemize}
	\item<+-> Data holder with set / get methods and no behaviour..
	\item<+-> Encapsulate fields, move methods and / or put things in a way that they can receive responsibility..
   \end{itemize}
\end{frame}

\subsection{Encapsulate field}
\begin{frame}
  \frametitle{Encapsulate field}
  \begin{itemize}
	\item<+-> Data hiding is a principal tenet of OO..because behavior should aways lie where data is!
	\item<+-> Making public fields permits data modification from objects different from data owner..(i.e. data is separated from behavior)
	\item<+-> When data and behavior are clustered togheter in a single point you can change behaviour in a simpler way.. 
	\item<+-> \textit{Encapsulate field} is the first baby step to keep information hiding.. other moves like \textit{Move Method} follow..
   \end{itemize}
\end{frame}

\section{Encapsulate collection}
\subsection{Encapsulate collection}
\begin{frame}
  \frametitle{Encapsulate collection}
  \begin{itemize}
	\item<+-> When a class contains collections, can give access to them throug setter or getter..
	\item<+-> Getters should return unmodifiable version of the collection to avoid clients to modify it.. the owner class can not be aware of changes
	\item<+-> Setters should be avoided. To access collection (i.e. add, remove and so on..) shoulb be demanded to the owner class.
   \end{itemize}
\end{frame}

\subsection{Remove setting method}
\begin{frame}
  \frametitle{Remove setting method}
  \begin{itemize}
	\item<+-> ``A field should be set at creation time and never altered!''
	\item<+-> ..declare field as final and init it at creation time.
	\item<+-> The message is clear to other programmers and you avoid unexpected behavior with separating data from behavior as public fields
   \end{itemize}
\end{frame}

\subsection{Hide Method}
\begin{frame}
  \frametitle{Hide Method}
  \begin{itemize}
	\item<+-> Relax visibility of a metod is a simple thing.. the other way around is more difficult (..nowaday we have automatic toold that can do this)
	\item<+-> If a method is not used hide it making private.. 
	\item<+-> This can be a first step to delete the whole method or to mitigate problems like direct access to data (i.e. hiding setter / getter)
   \end{itemize}
\end{frame}

\section{Refused Bequest}
\subsection{Refused Bequest}
\begin{frame}
  \frametitle{Refused Bequest}
  \begin{itemize}
	\item<+-> A child don't want behavior inherited from parent... the hierarchy is wrong
	\item<+-> You need to create a sibling and use \textit{Push Down Method}, \textit{Push Down Field} to move undesidered behavior
	\item<+-> In this way all the common behavior is the parent (.. and probably he is also abstract..)
   \end{itemize}
\end{frame}

\subsection{Push Down Method}
\begin{frame}
  \frametitle{Push Down Method}
  \begin{itemize}
	\item<+-> A behavior is relevant only to a specific subclass
	\item<+-> Move in the appropriate subclass
   \end{itemize}
\end{frame}

\subsection{Push Down Field}
\begin{frame}
  \frametitle{Push Down Field}
  \begin{itemize}
	\item<+-> A field is used only by some subclasses.
	\item<+-> Move the field to those subclasses
   \end{itemize}
\end{frame}

\section{Comments}
\subsection{Comments}
\begin{frame}
  \frametitle{Comments}
  \begin{itemize}
	\item<+-> A comment is often a deodorant to bad code..
	\item<+-> You need it when things aren't clear.. 
	\item<+-> Applying refactors like \textit{Extract Method}, \textit{Rename Method} or \textit{Introduce Assertion} make comment often superfluos
   \end{itemize}
\end{frame}

\subsection{Introduce Assertion}
\begin{frame}
  \frametitle{Introduce Assertion}
  \begin{itemize}
	\item<+-> To explicit an assumption in the code consider to introduce an assertion instead of comments..  
	\item<+-> Assertions in java are tricky..
   \end{itemize}
\end{frame}
\end{document}
