%
% -- Manlio Modugno

\documentclass{beamer} 
\usepackage{eulervm}
%\usepackage{booktabs}
\usepackage{listings}
\usepackage{bold-extra}
\usepackage{cancel}
\usepackage{fancybox}
\usepackage{soul}
\usepackage[english]{babel}
\usepackage[utf8]{inputenc}
\usepackage{hyperref}
\usepackage{amsmath}
%\hypersetup{colorlinks=true,urlcolor=blue}

\newcommand{\codefont}{\fontsize{6}{8}\selectfont}
\lstset{language=[Sharp]C, 
captionpos=b, 
frame=lines,
lineskip= 1pt, %space between lines
basicstyle=\codefont, 
keywordstyle=\color{blue}, 
commentstyle=\color{green}, 
stringstyle=\color{red}, 
numbers=left, 
numberstyle=\tiny, 
stepnumber=2,
numbersep=5pt,
breaklines=true, 
breakatwhitespace=false,
showstringspaces=false,
frame=single,
tabsize=2,
emph={double,bool,int,unsigned,char,true,false,void},
emphstyle=\color{blue},
emph={Assert,Test},
emphstyle=\color{red},
emph={[2]\using,\#define,\#ifdef,\#endif},
emphstyle={[2]\color{blue}}
}

\mode<presentation>
\definecolor{title_color}{RGB}{2,128,181} 
\usetheme{Ilmenau}
\usecolortheme[named=title_color]{structure}
\setbeamercolor{palette quaternary}{use=structure,fg=black,bg=white} %header footer color
\useoutertheme[subsection=false]{smoothbars}
\setbeamercovered{transparent}
\setbeamertemplate{navigation symbols}{}
\setbeamerfont{subsection in toc}{size=\scriptsize}

\title{Larman - OOA/D}
\author{Manlio Modugno}
\institute[GMTechnologies] 

\date[]{Use Cases (continue)}

\subject{}

\graphicspath{{img/}}
\pgfdeclareimage[height=0.6cm]{mfg-logo}{img/mfgLogo}
\logo{\pgfuseimage{mfg-logo}}

%
% Content start
%
\begin{document}
\begin{frame}
  \titlepage
\end{frame}

\begin{frame}
  \frametitle{Topics}
  \tableofcontents
\end{frame}


\section{Notation: What are Three Common Use Case Formats?}
\subsection{Notation: What are Three Common Use Case Formats?}
\begin{frame}
  \frametitle{Notation: What are Three Common Use CaseFormats?}
  \begin{itemize}
	\item<+-> \textbf{brief:} short one-paragraph summary, usually of the main success scenario
	\item<+-> \textbf{casual:} informal paragraphs that cover various scenarios
	\item<+-> \textbf{fully dressed:} all steps and variations written in detail
   \end{itemize}
\end{frame}

\section{Fully dressed template/sections}
\subsection{Fully dressed template/sections}
\begin{frame}
  \frametitle{Fully dressed template/sections}
  \begin{itemize}
	\item<+-> a common template was created by Cockburn..
   \end{itemize}
\end{frame}


\begin{frame}
  \frametitle{Fully dressed template/sections}
  \begin{itemize}
	\item<+-> Stakeholders and Interests $ \rightarrow $ `What should be in the use case?`
	\item<+-> That which satisfies all the stakeholders' interests
   \end{itemize}
\end{frame}

\subsection{Preface elements}
\begin{frame}
  \frametitle{Preface elements}
  \begin{itemize}
	\item<+-> \textbf{Scope:} bounds system(s) under design
	\item<+-> \textbf{Primary actor:} who calls a system services to fulfill a goal
	\item<+-> \textbf{Stakeholders and Interests:} suggests and bounds what the system must do..answers the question ``What should be in the use case?'' $ \rightarrow $ ``That which satisfies all the stakeholders' interests''. Helps in don't missing something/someone
   \end{itemize}
\end{frame}


\subsection{Other elements}
\begin{frame}
  \frametitle{Other elements}
  \begin{itemize}

	\item<+-> \textbf{Preconditions:} state what must always be true before a scenario is begun in the use case
	\item<+-> \textbf{Success guarantees (or postconditions):} state what must be true on successful completion of the use case
	\item<+-> \textbf{Preconditions:} describes a typical success path that satisfies the interests of the stakeholders
	\item<+-> \textbf{Extensions (or Alternate Flows):} indicate all the other scenarios or branches, both success and failure
   \end{itemize}
\end{frame}

\subsection{final elements}
\begin{frame}
  \frametitle{final elements}
  \begin{itemize}

	\item<+-> \textbf{Special Requirements:} non-functional requirement such as performance, reliability, usability..
	\item<+-> \textbf{Technology and Data Variations List:} tech variations related on \textit{how} something must be done such as i/o constraints..
   \end{itemize}
\end{frame}

\section{Conclusion}
\subsection{Congratulations: Use Cases are Written and Wrong (!)}
\begin{frame}
  \frametitle{Congratulations: Use Cases are Written and Wrong (!)}
  \begin{itemize}

	\item<+-> A long and detailed list of use cases doesn't give certainty about correctness of the system.. 
	\item<+-> Only code and tests (can) say what is true or not!
   \end{itemize}
\end{frame}


\section{Case Studies}
\subsection{Case One: The NextGen POS System}
\begin{frame}
  \frametitle{Case One: The NextGen POS System}
  \begin{itemize}
	\item<+-> A POS system is a computerized application used (in part) to record sales and handle payments
	\item<+-> It interfaces to various service applications
	\item<+-> These systems must be relatively fault-tolerant
	\item<+-> Supports various client-side terminals (web / pc / and so on..) 
	\item<+-> Must be flexible with respect to various clients with disparate needs in terms of business rule processing
   \end{itemize}
\end{frame}

\subsection{Case Two: The Monopoly Game System}
\begin{frame}
	\frametitle{Case Two: The Monopoly Game System}
  	\begin{itemize}
	\item<+-> OOA/D apply to very different problems.. Monopoly game
	\item<+-> Domain modeling, Object design and UML are still relevant and useful
	\item<+-> The software version of the game will run as a simulation
   \end{itemize}
\end{frame}

\section{Use Cases}
\subsection{Use Cases}
\begin{frame}
	\frametitle{Use Cases:}
	\begin{itemize}
	\item<+-> Are text stories used to discover and record requirements
	\end{itemize}
\end{frame}

\begin{frame}
	\frametitle{Example: Process Sale}
	\begin{center}
	A customer arrives at a checkout with items to purchase. The cashier uses
the POS system to record each purchased item. The system presents a running total and
line-item details. The customer enters payment information, which the system validates and
records. The system updates inventory. The customer receives a receipt from the system
and then leaves with the items
	\end{center}
\end{frame}

\begin{frame}
	\frametitle{Use Cases:}
	\begin{itemize}
	\item<+-> use cases are not diagrams, they are text
	\item<+-> target is to discover and record functional requirement that satisfy user's needs..
	\item<+-> ..it's not so simple..!
	\end{itemize}
\end{frame}

\subsection{What are Actors, Scenarios, and Use Cases?}
\begin{frame}
	\frametitle{What are Actors, Scenarios, and Use Cases?}
	\begin{itemize}
	\item<+-> \textbf{Actor:} is something with behavior.. person(role), computer system, ...
	\item<+-> \textbf{Scenario(use case instance):} specific sequence of actions and interactions between actors and the system, a particular story/path in using the system
	\item<+-> \textbf{Use case:} is a collection of related success and failure scenarios that describe
an actor using a system to support a goal 
	\end{itemize}
\end{frame}

\subsection{Use Cases and the Use-Case Model}
\begin{frame}
	\frametitle{Use Cases and the Use-Case Model}
	\begin{itemize}
		\item<+-> use-case modeling is an act of writing text, not drawing diagrams artifacts of systems
		\item<+-> ..can include \textit{optionally} diagrams
		\item<+-> There's no OOA/D in use case modeling.. but it's a key concept input to OOA/D
    \end{itemize}
\end{frame}

\subsection{Motivation: Why Use Cases?}
\begin{frame}
	\frametitle{Motivation: Why Use Cases?}
	\begin{itemize}
		\item<+-> Many ways exist to capture/describe goals..
		\item<+-> But a simpler written text is preferable, because customer can actively participate..  
		\item<+-> ...and this help in don't missing the point 
		\item<+-> Lack of user involvement in software projects is near the top of the list of reasons for project failure
		\item<+-> emphasize the user goals and perspective
		\item<+-> ...keep it simple (as everything!).. write text stories
    \end{itemize}
\end{frame}

\subsection{Definition: Are Use Cases Functional Requirements?} 
\begin{frame}
	\frametitle{Definition: Are Use Cases Functional Requirements?}
	\begin{itemize}
    \item<+-> Use cases are requirements, primarily functional or behavioral requirements that indicate what
the system will do
    \item<+-> define a contract of how a system will behave
    \end{itemize}
\end{frame}

\subsection{Definition: What are Three Kinds of Actors?}
\begin{frame}
	\frametitle{Three Ways to Apply UML}
	Various actors can exist in a use case, can be roles like people, orgs, software and machines 
	\begin{itemize}
		\item<+-> \textbf{Primary actor:} has user goals fulfilled through using services of the system 
		\item<+-> \textbf{Supporting actor:} provides a service to the system
		\item<+-> \textbf{Offstage actor:} has an interest in the behavior of the use case
    \end{itemize}
\end{frame}

\end{document}
