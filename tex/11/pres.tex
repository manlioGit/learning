%
% -- Manlio Modugno

\documentclass{beamer} 
\usepackage{eulervm}
%\usepackage{booktabs}
\usepackage{listings}
\usepackage{bold-extra}
\usepackage{cancel}
\usepackage{fancybox}
\usepackage{soul}
\usepackage[english]{babel}
\usepackage[utf8]{inputenc}
\usepackage{hyperref}
\usepackage{amsmath}
%\hypersetup{colorlinks=true,urlcolor=blue}

\newcommand{\codefont}{\fontsize{6}{8}\selectfont}
\lstset{language=[Sharp]C, 
captionpos=b, 
frame=lines,
lineskip= 1pt, %space between lines
basicstyle=\codefont, 
keywordstyle=\color{blue}, 
commentstyle=\color{green}, 
stringstyle=\color{red}, 
numbers=left, 
numberstyle=\tiny, 
stepnumber=2,
numbersep=5pt,
breaklines=true, 
breakatwhitespace=false,
showstringspaces=false,
frame=single,
tabsize=2,
emph={double,bool,int,unsigned,char,true,false,void},
emphstyle=\color{blue},
emph={Assert,Test},
emphstyle=\color{red},
emph={[2]\using,\#define,\#ifdef,\#endif},
emphstyle={[2]\color{blue}}
}


\mode<presentation>
\definecolor{title_color}{RGB}{2,128,181} 
\usetheme{Ilmenau}
\usecolortheme[named=title_color]{structure}
\setbeamercolor{palette quaternary}{use=structure,fg=black,bg=white} %header footer color
\useoutertheme[subsection=false]{smoothbars}
\setbeamercovered{transparent}
\setbeamertemplate{navigation symbols}{}
\setbeamerfont{subsection in toc}{size=\scriptsize}

\title{Bad Smells in Code / Refactorings - part2}
\author{Manlio Modugno}
\institute[GMTechnologies] 

\date[27.10.2016] 
{27.10.2016 - Bad Smells in Code - Bad Smells (continue)}

\subject{}

\graphicspath{{img/}}
\pgfdeclareimage[height=0.6cm]{mfg-logo}{img/mfgLogo}
\logo{\pgfuseimage{mfg-logo}}

%
% Content start
%
\begin{document}
\begin{frame}
  \titlepage
\end{frame}

\begin{frame}
  \frametitle{Argomenti Trattati}
  \tableofcontents
\end{frame}

\section{Bad Smells in Code}
\subsection{Data Clumps}
\begin{frame}
  \frametitle{Data Clumps}
  \begin{itemize}
	\item<+-> When you see a group of data passed around..
	\item<+-> Group them usign \textit{Extract Class} if they are fields..
	\item<+-> \textit{Introduce Parameter Object}, \textit{Preserve Whole Object} in method signatures...
  \end{itemize}
\end{frame}

\subsection{Primitive Obsession}
\begin{frame}
  \frametitle{Pull Up Method}
  \begin{itemize}
	\item<+-> When you have identical methods on subclasses...
	\item<+-> ..whenever there is duplication, you face the risk that an alteration to one will not be made to the other
	\item<+-> When there are similar methods.. start from generlize them (i.e. extract same behavior and parameterize)
  \end{itemize}
\end{frame}

\subsection{Switch Statements}
\begin{frame}
  \frametitle{Switch Statements}
  \begin{itemize}
	\item<+-> ..are sign of procedural code
	\item<+-> ..favor duplication (scattered around the code)
	\item<+-> polymorphism can tackle them.. (pay attention to overkill..)
  \end{itemize}
\end{frame}


\subsection{Parallel Inheritance Hierarchies}
\begin{frame}
  \frametitle{Parallel Inheritance Hierarchies}
  \begin{itemize}
	\item<+-> A special case of \textit{Shotgun Surgery}..
	\item<+-> ...you see it every time a subclass borns, another with same prefix comes along 
	\item<+-> ... referencing instances of one hierarchy to another one helps eliminating referring classes
  \end{itemize}
\end{frame}

\section{Refectorings}
\subsection{Replace Data Value with Object}
\begin{frame}
  \frametitle{Replace Data Value with Object}
  \begin{itemize}
	\item<+-> Representation of 'simple' concepts made with 'simple' data items..
	\item<+-> ...during development process 'simple' data items can't represent concept anymore..
	\item<+-> ... covert data itme into an object.
  \end{itemize}
\end{frame}

\subsection{Replace Type Code with Class}
\begin{frame}
  \frametitle{Replace Type Code with Class}
  \begin{itemize}
	\item<+-> Enumerations are alias to numbers..compiler checks number not symbolic name(true also with last jdks?)
	\item<+-> Replacing with a class permits compile time check
	\item<+-> Use when you have a type code that does not affect behavior.
  \end{itemize}
\end{frame}

\subsection{Replace Type Code with Subclasses}
\begin{frame}
  \frametitle{Replace Type Code with Subclasses}
  \begin{itemize}
	\item<+-> When you have type code that affect behavior..
	\item<+-> Usually in conditionals.. type code satisfy conditon $\rightarrow$ do something..
	\item<+-> This refactor enables \textit{Replace Conditional With Polymorphism}
	\item<+-> Create a new subclass for each type.. and you move knowledge of the variant behavior from clients of the class to the class itself 
  \end{itemize}
\end{frame}

\subsection{Replace Type Code with State/Strategy}
\begin{frame}
  \frametitle{Replace Type Code with State/Strategy}
  \begin{itemize}
	\item<+-> When you have a type code that changes during object life time or another reason prevents subclassing
	\item<+-> Use state / strategy pattern (GoF). Algorithm $\rightarrow$ Strategy.. changing state on object $\rightarrow$state
  \end{itemize}
\end{frame}
  
\subsection{Replace Array with Object}
\begin{frame}
  \frametitle{Replace Array with Object}
  \begin{itemize}
	\item<+-> When you have data structure in which certain elements mean different things..
	\item<+-> ..substitue with an object.
	\item<+-> Data structure shoul be used to keep organized data..
  \end{itemize}  
\end{frame}

\end{document}
