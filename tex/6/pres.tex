%
% -- Manlio Modugno

\documentclass{beamer} 
\usepackage{eulervm}
%\usepackage{booktabs}
\usepackage{listings}
\usepackage{bold-extra}
\usepackage{cancel}
\usepackage{fancybox}
\usepackage{soul}
\usepackage[english]{babel}
\usepackage[utf8]{inputenc}
\usepackage{hyperref}
\usepackage{amsmath}
%\hypersetup{colorlinks=true,urlcolor=blue}

\newcommand{\codefont}{\fontsize{6}{8}\selectfont}
\lstset{language=[Sharp]C, 
captionpos=b, 
frame=lines,
lineskip= 1pt, %space between lines
basicstyle=\codefont, 
keywordstyle=\color{blue}, 
commentstyle=\color{green}, 
stringstyle=\color{red}, 
numbers=left, 
numberstyle=\tiny, 
stepnumber=2,
numbersep=5pt,
breaklines=true, 
breakatwhitespace=false,
showstringspaces=false,
frame=single,
tabsize=2,
emph={double,bool,int,unsigned,char,true,false,void},
emphstyle=\color{blue},
emph={Assert,Test},
emphstyle=\color{red},
emph={[2]\using,\#define,\#ifdef,\#endif},
emphstyle={[2]\color{blue}}
}


\mode<presentation>
\definecolor{title_color}{RGB}{2,128,181} 
\usetheme{Ilmenau}
\usecolortheme[named=title_color]{structure}
\setbeamercolor{palette quaternary}{use=structure,fg=black,bg=white} %header footer color
\useoutertheme[subsection=false]{smoothbars}
\setbeamercovered{transparent}
\setbeamertemplate{navigation symbols}{}
\setbeamerfont{subsection in toc}{size=\scriptsize}

\title{Why Extends is Evil}
\author{Manlio Modugno}
\institute[GMTechnologies] 

\date[15.09.2016] 
{15.09.2016 - Principles in Refactoring}

\subject{}

\graphicspath{{img/}}
\pgfdeclareimage[height=0.6cm]{mfg-logo}{img/mfgLogo}
\logo{\pgfuseimage{mfg-logo}}

%
% Content start
%
\begin{document}
\begin{frame}
  \titlepage
\end{frame}

\begin{frame}
  \frametitle{Argomenti Trattati}
  \tableofcontents
\end{frame}

\section{Principles in Refactoring}
\subsection{Defining Refactoring}
\begin{frame}
  \frametitle{Defining Refactoring (Definition)}
  \begin{itemize}
	\item<+-> as noun: ``a change made to the internal structure of software to make it 
\textbf{easier to understand} and \textbf{cheaper to modify} without changing its observable behavior''
	\item<+-> as verb: ``to restructure software by applying a series of refactorings without changing its observable behavior''
  \end{itemize}
\end{frame}

\begin{frame}
  \frametitle{The Two Hats}
  \begin{itemize}
  		\item<+-> While you develop, you divide your time between adding functions and refactoring
		\item<+-> When you add function, you shouldn't be changing existing code; you are just adding 
new capabilities
		\item<+-> When you refactor, you make a point of not adding function; you only restructure the code
		\item<+-> ...as you develop you swap hat frequently
		\item<+-> Refactor it's not the cure for all software ills.. however It's a valuable tool for several things.. 		
  \end{itemize}
\end{frame}

\section{Why should you refactor?}
\subsection{Refactoring Improves the Design of Software}
\begin{frame}
  \frametitle{Why should you refactor? \textbf{Refactoring Improves the Design of Software}} 
  \begin{itemize}
  		\item<+-> Without refactor, as people change code, the code lose its stucture..
		\item<+-> ..it becomes harder to see the design by reading the code.. 
		\item<+-> .. so it becomes harder to preserve it.. 
		\item<+-> ...a bad loop that brings code to collapse. 
  \end{itemize}
\end{frame}

\begin{frame}
  \frametitle{Why should you refactor? \textbf{Refactoring Improves the Design of Software}} 
  \begin{itemize}
  		\item<+-> \textbf{Poorly designed code usually takes more code to do the same things}
		\item<+-> ..design improvement strictly related to duplication removal!
		\item<+-> \textbf{Reducing amount of the code} facilitates code modification
		\item<+-> ``The more code there is, the harder it is to modify correctly. There's more code to understand ''
		\item<+-> When code says everything once and only once we have essence of good design
  \end{itemize}
\end{frame}

\subsection{Refactoring Makes Software Easier to Understand}
\begin{frame}
  \frametitle{Why should you refactor? \textbf{Refactoring Makes Software Easier to Understand}} 
  \begin{itemize}
  		\item<+-> Programming is like a conversation between you and a computer..
		\item<+-> ..you give an high-level specification to a pc and It does what you say (hopefully!)
		\item<+-> ..but your code is also read by other people (..and from you in different points in time..)
		\item<+-> when we write code we must think to future developers that will read that code!
		\item<+-> Refactoring helps to keep code more readable..
  \end{itemize}
\end{frame}

\subsection{Refactoring Makes Software Easier to Understand}
\begin{frame}
  \frametitle{Why should you refactor? \textbf{Refactoring Helps You Find Bugs}} 
  \begin{itemize}
  		\item<+-> 
		
  \end{itemize}
\end{frame}

\begin{frame}[containsverbatim]
	\frametitle{Losing flexibility}
	A core concept to achieve flexibility is \textit{program to interfaces} \\
	Consider following example where concrete class is used:
	\begin{lstlisting}
f()
{   LinkedList list = new LinkedList();
    //...
    g( list );
}
g( LinkedList list )
{
    list.add( ... );
    g2( list )
}
...
\end{lstlisting}
\end{frame}

\begin{frame}[containsverbatim]
	\frametitle{Losing flexibility}
	If \textbf{LinkedList} no more fits and we need an HashSet we need   \\
	to change \underline{every} point in the code. Using interface we change just one point \\
	\begin{lstlisting}
f()
{   Collection list = new LinkedList();
    //...
    g( list );
}
g( Collection list )
{
    list.add( ... );
    g2( list )
}
...
\end{lstlisting}
\end{frame}

\subsection{Coupling}
\begin{frame}
  \frametitle{Coupling}
  \begin{itemize}
  		\item<+-> Inheritance has strong coupling (i.e. ``undesirable reliance of one part of a program on another part'')
		\item<+-> \textbf{Global variables(!)} are strongest example of coupling! 
		\item<+-> As designer, we should strive to minimize coupling relationships.
		\item<+-> ..of course we must have some loose coupling.. but
		\item<+-> ``the most important (OO rule) is that the implementation of an object should be completely hidden from the objects that use it'' $\Rightarrow$ to minimize coupling
		\item<+-> Direct exp: ``I don't have time to rewrite programs, so I follow the rules. My concern is entirely practical. I have no interest in purity for the sake of purity.''
  \end{itemize}
\end{frame}

\begin{frame}
  \frametitle{the fragile base-class problem}
  \begin{itemize}
  \item<+-> A change in a base class impacts all the hierarchy! 
  \item<+-> We should check every class coupled with the base one..
  \item<+-> ..and every class that uses Base o derived!
  \end{itemize}
\end{frame}


\begin{frame}[containsverbatim]
	\frametitle{the fragile base-class problem}
	Stack implemented inherinting from ArrayList   \\
	A user using any derived method from ArrayList can have big surprises! (i.e. clear)\\
	\begin{lstlisting}
class Stack extends ArrayList
{   private int stack_pointer = 0;
    public void push( Object article )
    {   add( stack_pointer++, article );
    }
    public Object pop()
    {   return remove( --stack_pointer );
    }
    public void push_many( Object[] articles )
    {   for( int i = 0; i < articles.length; ++i )
            push( articles[i] );
    }
}
...
\end{lstlisting}
\end{frame}

\begin{frame}
  \frametitle{the fragile base-class problem}
  \begin{itemize}
  \item<+-> A solution is to redefine all the inherited methods..
  \item<+-> ..pain in the ass .. maybe you should use an interface!
  \item<+-> ..and if you throw exceptions to not supported method you break LSP and you move error prone behaviour
  from compile time to run time! 
  \end{itemize}
\end{frame}

\begin{frame}[containsverbatim]
	\frametitle{the fragile base-class problem}
	A better solution is to encapsulate the data structure   \\
	\begin{lstlisting}
class Stack
{   private int stack_pointer = 0;
    private ArrayList the_data = new ArrayList();
    public void push( Object article )
    {   the_data.add( stack_pointer++, article );
    }
    public Object pop()
    {   return the_data.remove( --stack_pointer );
    }
    public void push_many( Object[] articles )
    {   for( int i = 0; i < o.length; ++i )
            push( articles[i] );
    }
}
...
\end{lstlisting}
\end{frame}

\begin{frame}[containsverbatim]
	\frametitle{the fragile base-class problem}
	Very good!.. but one day someone(even yourself) create a variant of the Stack   \\
	\begin{lstlisting}
class Monitorable_stack extends Stack
{
    private int high_water_mark = 0;
    private int current_size;
    public void push( Object article )
    {   if( ++current_size > high_water_mark )
            high_water_mark = current_size;
        super.push(article);
    }
    
    public Object pop()
    {   --current_size;
        return super.pop();
    }
    public int maximum_size_so_far()
    {   return high_water_mark;
    }
}
...
\end{lstlisting}
\end{frame}

\begin{frame}[containsverbatim]
	\frametitle{the fragile base-class problem}
	And one fine day someone(again even yourself).. modifies internal implementation of the base class..   \\
	\begin{lstlisting}
class Stack
{   private int stack_pointer = -1;
    private Object[] stack = new Object[1000];
    public void push( Object article )
    {   assert stack_pointer < stack.length;
        stack[ ++stack_pointer ] = article;
    }
    public Object pop()
    {   assert stack_pointer >= 0;
        return stack[ stack_pointer-- ];
    }
    public void push_many( Object[] articles )
    {   assert (stack_pointer + articles.length) < stack.length;
        System.arraycopy(articles, 0, stack, stack_pointer+1,
                                                articles.length);
        stack_pointer += articles.length;
    }
}
\end{lstlisting}
\end{frame}

\begin{frame}
  \frametitle{the fragile base-class problem}
  \begin{itemize}
  \item<+-> pushmany no more call push to accomplish his task... we have a problem!
  \item<+-> Derived class Monitorablestack no more track water marker..
  \item<+-> ..there is no solution to these kind of problems...  
  \end{itemize}
\end{frame}

\begin{frame}[containsverbatim]
	\frametitle{the fragile base-class problem}
	..but you can avoide them using interfaces!   \\
	\begin{lstlisting}
interface Stack{
	void push( Object o );
	Object pop();
	void push_many( Object[] source );
}

class Simple_stack implements Stack{   
	private int stack_pointer = -1;
	private Object[] stack = new Object[1000];
	...
}
class Monitorable_Stack implements Stack{
    private int high_water_mark = 0;
    private int current_size;
    Simple_stack stack = new Simple_stack();
    ...
}

\end{lstlisting}
\end{frame}
\subsection{Frameworks}
\begin{frame}
  \frametitle{Frameworks}
  \textbf{Pay attention to frameworks.. they try to realize the \textit{final solution}}..  maybe they are not so good for your needs (even when they are member of nobility..) \\

\end{frame}

\end{document}