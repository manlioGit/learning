%
% -- Manlio Modugno

\documentclass{beamer} 
\usepackage{eulervm}
%\usepackage{booktabs}
\usepackage{listings}
\usepackage{bold-extra}
\usepackage{cancel}
\usepackage{fancybox}
\usepackage{soul}
\usepackage[english]{babel}
\usepackage[utf8]{inputenc}
\usepackage{hyperref}
\usepackage{amsmath}
%\hypersetup{colorlinks=true,urlcolor=blue}

\newcommand{\codefont}{\fontsize{6}{8}\selectfont}
\lstset{language=[Sharp]C, 
captionpos=b, 
frame=lines,
lineskip= 1pt, %space between lines
basicstyle=\codefont, 
keywordstyle=\color{blue}, 
commentstyle=\color{green}, 
stringstyle=\color{red}, 
numbers=left, 
numberstyle=\tiny, 
stepnumber=2,
numbersep=5pt,
breaklines=true, 
breakatwhitespace=false,
showstringspaces=false,
frame=single,
tabsize=2,
emph={double,bool,int,unsigned,char,true,false,void},
emphstyle=\color{blue},
emph={Assert,Test},
emphstyle=\color{red},
emph={[2]\using,\#define,\#ifdef,\#endif},
emphstyle={[2]\color{blue}}
}


\mode<presentation>
\definecolor{title_color}{RGB}{2,128,181} 
\usetheme{Ilmenau}
\usecolortheme[named=title_color]{structure}
\setbeamercolor{palette quaternary}{use=structure,fg=black,bg=white} %header footer color
\useoutertheme[subsection=false]{smoothbars}
\setbeamercovered{transparent}
\setbeamertemplate{navigation symbols}{}
\setbeamerfont{subsection in toc}{size=\scriptsize}

\title{Principles in Refactoring - part 2}
\author{Manlio Modugno}
\institute[GMTechnologies] 

\date[22.09.2016] 
{22.09.2016 - Principles in Refactoring - part 2}

\subject{}

\graphicspath{{img/}}
\pgfdeclareimage[height=0.6cm]{mfg-logo}{img/mfgLogo}
\logo{\pgfuseimage{mfg-logo}}

%
% Content start
%
\begin{document}
\begin{frame}
  \titlepage
\end{frame}

\begin{frame}
  \frametitle{Argomenti Trattati}
  \tableofcontents
\end{frame}

\section{Principles in Refactoring- part 2}
\subsection{What do I tell my manager?}
\begin{frame}
  \frametitle{What do I tell my manager?}
  \begin{itemize}
	\item<+-> A tech savvy manager understands principles.. and he's the one who starts it perhaps!
	\item<+-> ..a manager more driven by schedule than quality... don't tell...(I don't totally agree..) 
	\item<+-> ..but if the fastest way to getting things done is to refactor.. than I refactor!
  \end{itemize}
\end{frame}

\subsection{Indirection and refactoring}
\begin{frame}
  \frametitle{Indirection and refactoring}
  \begin{itemize}
	\item<+-> Most refactorings introduces a lot of indirection, it tends to break big objects/methods into several smaller ones 
	\item<+-> Pay attention to multi layer indirection!
  \end{itemize}
\end{frame}

\section{Problems With Refactoring}
\subsection{Intro}
\begin{frame}
  \frametitle{Problems With Refactoring} 
  \begin{itemize}
  		\item<+-> ...don't fall in the white rabbit's hole!..
  		\item<+-> .. always evaluate deadlines.. 
  		\item<+-> ..make more little refactors to accomplish a bigger one later..
  \end{itemize}
\end{frame}

\subsection{Databases}
\begin{frame}
  \frametitle{Databases} 
  \begin{itemize}
  		\item<+-> DB are (usually) difficult to change and migrate (..solutions exist nowadays..)
  		\item<+-> Layer between application logic and data can be a solution..
  \end{itemize}
\end{frame}

\subsection{Interfaces}
\begin{frame}
  \frametitle{Interfaces} 
  \begin{itemize}
  		\item<+-> When you change an interface anything can happen..
  		\item<+-> A \textit{rename method} can't be a problem if you can acess all the code.. otherwise you have a ``published interface''..
  		\item<+-> ..in this case the old method call the new one (without duplicating it!) and add a deprecation marker
  		\item<+-> When you can.. don't publish interfaces! (..and relax ownership policies on the code..)
  \end{itemize}
\end{frame}

\subsection{Design Changes That Are Difficult to Refactor}
\begin{frame}
  \frametitle{Design Changes That Are Difficult to Refactor} 
  \begin{itemize}
  		\item<+-> Are some design decisions so central that you cannot count on refactoring to change your mind later?
  		\item<+-> Probably you can count ... but crying a lot..
 
  		\item<+-> As a rule of thumb pick up the ``simplest solution''..
  \end{itemize}
\end{frame}

\subsection{When Shouldn't You Refactor?}
\begin{frame}
  \frametitle{When Shouldn't You Refactor?} 
  \begin{itemize}
  		\item<+-> When it's better to rewrite everyting from scratch!
  		\item<+-> To refactor you should have code that work mostly correctly..
  		\item<+-> With core legay systems refactor (i.e. susstitute) components one at a time..
  		\item<+-> When you're close to a deadline.. unnfinished refactors can be worst that the previuos state..!..
  		\item<+-> ..but remember to pay your debts!
  		\item<+-> ``Not having enough time usually is a sign that you need to do some refactoring''
  \end{itemize}
\end{frame}

\section{Refactoring and Design}
\subsection{intro}
\begin{frame}
  \frametitle{Refactoring and Design} 
  \begin{itemize}
  		\item<+-> Design is programming!
  		\item<+-> Refactoring can be an alternative to upfront design.. 
  		\item<+-> ..you don't need to find the \textit{right} solution.. you find the solution the continuosly adapt! (the reasonable one!)
  		\item<+-> ..as you dig into the problem, you better understand it.. so you adapt
  		\item<+-> ..so \textbf{keep your design as simple as possible!}
  \end{itemize}
\end{frame}  
  
  \begin{frame}
  \frametitle{Refactoring and Design} 
  \begin{itemize}
  		\item<+-> Finding flexible solutions that support change can be really expensive.. 
  		\item<+-> ..and generally they are really complex..
  		\item<+-> ..maybe you have software that can flex easily.. but you pay in maintainning it 
  		\item<+-> Refactoing can lead to \textbf{simpler} designs without sacrificing flexibility..
  		\item<+-> To summarize.. again.. \textbf{build the simplest thing that can possibly work!}
  \end{itemize}
\end{frame}

\section{Refactoring and Performance}
\subsection{intro}
\begin{frame}
  \frametitle{Refactoring and Performance} 
  \begin{itemize}
  		\item<+-> Common concern while coding..
  		\item<+-> Prefer understanding over performance..(in a reasonable way!)
  		\item<+-> A simple design can be easily tuned later..
  		\item<+-> ..consider context: real time applications and so on..
  		\item<+-> In optimizing time we should consider time to understand a piece of code also..
  		\item<+-> When you want to optimize.. don't speculate but use a profiler!
  		\item<+-> To summarize: \textbf{No premature optimization, but late optimize}..
  		\item<+-> ..to go faster.. refactor!
  \end{itemize}
\end{frame}

\end{document}