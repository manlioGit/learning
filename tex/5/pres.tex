%
% -- Manlio Modugno

\documentclass{beamer} 
\usepackage{eulervm}
%\usepackage{booktabs}
\usepackage{listings}
\usepackage{bold-extra}
\usepackage{cancel}
\usepackage{fancybox}
\usepackage{soul}
\usepackage[english]{babel}
\usepackage[utf8]{inputenc}
\usepackage{hyperref}
\usepackage{amsmath}
%\hypersetup{colorlinks=true,urlcolor=blue}

\newcommand{\codefont}{\fontsize{6}{8}\selectfont}
\lstset{language=[Sharp]C, 
captionpos=b, 
frame=lines,
lineskip= 1pt, %space between lines
basicstyle=\codefont, 
keywordstyle=\color{blue}, 
commentstyle=\color{green}, 
stringstyle=\color{red}, 
numbers=left, 
numberstyle=\tiny, 
stepnumber=2,
numbersep=5pt,
breaklines=true, 
breakatwhitespace=false,
showstringspaces=false,
frame=single,
tabsize=2,
emph={double,bool,int,unsigned,char,true,false,void},
emphstyle=\color{blue},
emph={Assert,Test},
emphstyle=\color{red},
emph={[2]\using,\#define,\#ifdef,\#endif},
emphstyle={[2]\color{blue}}
}


\mode<presentation>
\definecolor{title_color}{RGB}{2,128,181} 
\usetheme{Ilmenau}
\usecolortheme[named=title_color]{structure}
\setbeamercolor{palette quaternary}{use=structure,fg=black,bg=white} %header footer color
\useoutertheme[subsection=false]{smoothbars}
\setbeamercovered{transparent}
\setbeamertemplate{navigation symbols}{}
\setbeamerfont{subsection in toc}{size=\scriptsize}

\title{Why Extends is Evil}
\author{Manlio Modugno}
\institute[GMTechnologies] 

\date[29.07.2016] 
{29.07.2016 - Why Extends is Evil}

\subject{}

\graphicspath{{img/}}
\pgfdeclareimage[height=0.6cm]{mfg-logo}{img/mfgLogo}
\logo{\pgfuseimage{mfg-logo}}

%
% Content start
%
\begin{document}
\begin{frame}
  \titlepage
\end{frame}

\begin{frame}
  \frametitle{Argomenti Trattati}
  \tableofcontents
\end{frame}

\section{Why Extends is Evil}
\subsection{abstract}
\begin{frame}
  \frametitle{abstract}
  \begin{itemize}
	\item<+-> Avoid \textit{extends} whenever possible... (or at least count untill 100 before to use it!)
	\item<+-> Use interface inheritance! (i.e. program to interfaces)
	\item<+-> Good designers write most of their code in terms of interfaces, not concrete base classes!
  \end{itemize}
\end{frame}

\subsection{Losing flexibility}
\begin{frame}
  \frametitle{Losing flexibility}
  \begin{itemize}
  		\item<+-> Explicit use of concrete class names locks you(r mind) into specific implementations
		\item<+-> \textbf{Parallel Design:} start programming \underline{before} before fully specify the problem, a \textit{design} (remember? design = code!) should not be necessarily complete before to start programming.
		\item<+-> To apply parallel design we must have flexibility: \textbf{write your code in such a way that you can incorporate newly discovered requirements into the existing code as painlessly as possible}
		\item<+-> Rather than implement features you might need, you implement only the features you definitely need, but in a way that accommodates change
  \end{itemize}
\end{frame}

\begin{frame}[containsverbatim]
	\frametitle{Losing flexibility}
	A core concept to achieve flexibility is \textit{program to interfaces} \\
	Consider following example where concrete class is used:
	\begin{lstlisting}
f()
{   LinkedList list = new LinkedList();
    //...
    g( list );
}
g( LinkedList list )
{
    list.add( ... );
    g2( list )
}
...
\end{lstlisting}
\end{frame}

\begin{frame}[containsverbatim]
	\frametitle{Losing flexibility}
	If \textbf{LinkedList} no more fits and we need an HashSet we need   \\
	to change \underline{every} point in the code. Using interface we change just one point \\
	\begin{lstlisting}
f()
{   Collection list = new LinkedList();
    //...
    g( list );
}
g( Collection list )
{
    list.add( ... );
    g2( list )
}
...
\end{lstlisting}
\end{frame}

\subsection{Coupling}
\begin{frame}
  \frametitle{Coupling}
%  \begin{itemize}
%  		\item<+-> Explicit use of concrete class names locks you(r mind) into specific implementations
%		\item<+-> \textbf{Parallel Design:} start programming \underline{before} before fully specify the problem, a \textit{design} (remember? design = code!) should not be necessarily complete before to start programming.
%		\item<+-> To apply parallel design we must have flexibility: \textbf{write your code in such a way that you can incorporate newly discovered requirements into the existing code as painlessly as possible}
%		\item<+-> Rather than implement features you might need, you implement only the features you definitely need, but in a way that accommodates change
%  \end{itemize}
\end{frame}

\end{document}