%
% -- Manlio Modugno

\documentclass{beamer} 
\usepackage{eulervm}
%\usepackage{booktabs}
\usepackage{listings}
\usepackage{bold-extra}
\usepackage{cancel}
\usepackage{fancybox}
\usepackage{soul}
\usepackage[english]{babel}
\usepackage[utf8]{inputenc}
\usepackage{hyperref}
\usepackage{amsmath}
%\hypersetup{colorlinks=true,urlcolor=blue}

\newcommand{\codefont}{\fontsize{6}{8}\selectfont}
\lstset{language=[Sharp]C, 
captionpos=b, 
frame=lines,
lineskip= 1pt, %space between lines
basicstyle=\codefont, 
keywordstyle=\color{blue}, 
commentstyle=\color{green}, 
stringstyle=\color{red}, 
numbers=left, 
numberstyle=\tiny, 
stepnumber=2,
numbersep=5pt,
breaklines=true, 
breakatwhitespace=false,
showstringspaces=false,
frame=single,
tabsize=2,
emph={double,bool,int,unsigned,char,true,false,void},
emphstyle=\color{blue},
emph={Assert,Test},
emphstyle=\color{red},
emph={[2]\using,\#define,\#ifdef,\#endif},
emphstyle={[2]\color{blue}}
}


\mode<presentation>
\definecolor{title_color}{RGB}{2,128,181} 
\usetheme{Ilmenau}
\usecolortheme[named=title_color]{structure}
\setbeamercolor{palette quaternary}{use=structure,fg=black,bg=white} %header footer color
\useoutertheme[subsection=false]{smoothbars}
\setbeamercovered{transparent}
\setbeamertemplate{navigation symbols}{}
\setbeamerfont{subsection in toc}{size=\scriptsize}

\title{Bad Smells in Code / Rafactorings}
\author{Manlio Modugno}
\institute[GMTechnologies] 

\date[06.10.2016] 
{06.10.2016 - Bad Smells in Code / Rafactorings}

\subject{}

\graphicspath{{img/}}
\pgfdeclareimage[height=0.6cm]{mfg-logo}{img/mfgLogo}
\logo{\pgfuseimage{mfg-logo}}

%
% Content start
%
\begin{document}
\begin{frame}
  \titlepage
\end{frame}

\begin{frame}
  \frametitle{Argomenti Trattati}
  \tableofcontents
\end{frame}

\section{Refactorings}
\subsection{Extract Method}
\begin{frame}
  \frametitle{Extract Method}
  \begin{itemize}
	\item<+-> Apply on too long methods (consider semantic distance also: method name <-> method body) or to code with comments
	\item<+-> Prefer short well named methods..
	\item<+-> ..a litlle (well cohesive) method it's probably reused by other methods
    \item<+-> ..and ehance readibilty
  \end{itemize}
\end{frame}

\subsection{Duplicated Code}
\begin{frame}
  \frametitle{Duplicated Code}
  \begin{itemize}
	\item<+-> Number one in smell parade! Duplication destroy resilience
	\item<+-> ``simple'' duplication is quite easy to remove using refator like \textit{Extract Method}, etc..
	\item<+-> More difficult when we have conceptual duplication like different ways to access data layer, classes that mimic data structure already existent, etc..
  \end{itemize}
\end{frame}

\subsection{Long Method}
\begin{frame}
  \frametitle{Long Method}
  \begin{itemize}
	\item<+-> Keep your methods short!.. your code live longer.. with right indirection level 
	\item<+-> A long procedure is hard to read / undestand...
	\item<+-> With cohesive (little) methods \underline{\textbf{with good naming}} you don't need to look into to grasp intent...
	\item<+-> Decompose methods in a aggressive way (..but keep always good sense).. even to substitute a single line of code if new method name ehnance unserstandibility!
	\item<+-> Fundmental metric to consider is not only the phisical length of the method, but the \textbf{semantic distance} between what the method does and how it does it.
  \end{itemize}
\end{frame}


\begin{frame}
  \frametitle{Long Method - refactor}
  \begin{itemize}
	\item<+-> To reduce a method usually apply \textit{Extract Method} to parts of method that ``go nicely togheter'' (i.e. cohesive parts) 
	\item<+-> With methods with a lot of parameters (i.e. Long Parameter List) use \textit{Replace Temp With Query}, \textit{Introduce Parameter Object}, \textit{Preserve Whole Object} or \textit{Replace Method With Method Object} to avoid to pass around parameters to extracted methods
	\item<+-> When you see a comment there's probably an high semantic distance.. so \textit{Extract Method}  
	\item<+-> Conditional and loops are also a good sign to execute extraction.. use \textit{Decompose Conditional} or extract loop in its own new method
  \end{itemize}
\end{frame}

\subsection{Large Class}
\begin{frame}
  \frametitle{Large Class}
  \begin{itemize}
	\item<+-> A class is trying to do too much.. often has a lot of instance variables.. duplication is probably behind the corner 
	\item<+-> Apply \textit{Extract Class} or \textit{Extract Subclass} to a subset of instance variables in cohesive way
	\item<+-> A class with too much code support duplication, chaos and death!
	\item<+-> Another approach to decompose huge classes is to use \textit{Extract Interface} looking at clients' use
  \end{itemize}
\end{frame}
  
\subsection{Long Parameter List}
\begin{frame}
  \frametitle{Long Parameter List}
  \begin{itemize}
	\item<+-> With an OO lang you can pass to a method enough to satisfy everything it needs.. no more km long paramter list 
	\item<+-> Apply \textit{Raplace Parameter with Method}, \textit{Preserve Whole Object} or \textit{Introduce Parameter Object} to reduce amount of paraters passed around
	\item<+-> An exception is when you don't want to increase coupling between caller and callee...
  \end{itemize}
\end{frame}

\subsection{Divergent Change}
\begin{frame}
  \frametitle{Divergent Change}
  \begin{itemize}
	\item<+-> When making a change we want to jump to single simple clear point and make the change
	\item<+-> Divergent change occurs when one class is commonly changed in diffrent ways for different reasons..
	\item<+-> ..in this case you'd prefer to have more object instead of one..use \textit{Extract Class}
	\item<+-> Any change to handle a variation should change a single class
  \end{itemize}
\end{frame}

\subsection{Shotgun Surgery}
\begin{frame}
  \frametitle{Shotgun Surgery}
  \begin{itemize}
	\item<+-> Is the opposite of \textit{Divergent Change}.. when you have to make a lot of little changes to a lot of different classes
	\item<+-> Use \textit{Move Method}, \textit{Move Field} or \textit{Inline Class} to put all the changes into a single class
	\item<+-> Divergent change: a single class suffers of many changes 
	\item<+-> Shotgun surgery: a single change alters many classes
  \end{itemize}
\end{frame}

\subsection{Feature Envy}
\begin{frame}
  \frametitle{Feature Envy}
  \begin{itemize}
	\item<+-> Whole point of objects: a technique to package data with the processes used on that data!
	\item<+-> When you see an object more interested in components of another class other than its own properties / methods
	\item<+-> Use \textit{Move Method}, \textit{Extract Method} to give a new proper house
	\item<+-> In a mixed situation move method to the class that has most of the data used.
	\item<+-> Exceptions exist.. as rule of thumb .. put things together that change togheter.
  \end{itemize}
\end{frame}

\end{document}