%
% -- Manlio Modugno

\documentclass{beamer} 
\usepackage{eulervm}
%\usepackage{booktabs}
\usepackage{listings}
\usepackage{bold-extra}
\usepackage{cancel}
\usepackage{fancybox}
\usepackage{soul}
\usepackage[english]{babel}
\usepackage[utf8]{inputenc}
\usepackage{hyperref}
\usepackage{amsmath}
%\hypersetup{colorlinks=true,urlcolor=blue}

\newcommand{\codefont}{\fontsize{6}{8}\selectfont}
\lstset{language=[Sharp]C, 
captionpos=b, 
frame=lines,
lineskip= 1pt, %space between lines
basicstyle=\codefont, 
keywordstyle=\color{blue}, 
commentstyle=\color{green}, 
stringstyle=\color{red}, 
numbers=left, 
numberstyle=\tiny, 
stepnumber=2,
numbersep=5pt,
breaklines=true, 
breakatwhitespace=false,
showstringspaces=false,
frame=single,
tabsize=2,
emph={double,bool,int,unsigned,char,true,false,void},
emphstyle=\color{blue},
emph={Assert,Test},
emphstyle=\color{red},
emph={[2]\using,\#define,\#ifdef,\#endif},
emphstyle={[2]\color{blue}}
}


\mode<presentation>
\definecolor{title_color}{RGB}{2,128,181} 
\usetheme{Ilmenau}
\usecolortheme[named=title_color]{structure}
\setbeamercolor{palette quaternary}{use=structure,fg=black,bg=white} %header footer color
\useoutertheme[subsection=false]{smoothbars}
\setbeamercovered{transparent}
\setbeamertemplate{navigation symbols}{}
\setbeamerfont{subsection in toc}{size=\scriptsize}

\title{Bad Smells in Code / Rafactorings}
\author{Manlio Modugno}
\institute[GMTechnologies] 

\date[06.10.2016] 
{06.10.2016 - Bad Smells in Code / Rafactorings}

\subject{}

\graphicspath{{img/}}
\pgfdeclareimage[height=0.6cm]{mfg-logo}{img/mfgLogo}
\logo{\pgfuseimage{mfg-logo}}

%
% Content start
%
\begin{document}
\begin{frame}
  \titlepage
\end{frame}

\begin{frame}
  \frametitle{Argomenti Trattati}
  \tableofcontents
\end{frame}

\section{Refactorings}
\subsection{Extract Method}
\begin{frame}
  \frametitle{Extract Method}
  \begin{itemize}
	\item<+-> Apply on too long methods (consider semantic distance also: method name <-> method body) or to code with comments
	\item<+-> Prefer short well named methods..
	\item<+-> ..a litlle (well cohesive) method it's probably reused by other methods
    \item<+-> ..and ehance readibilty
  \end{itemize}
\end{frame}

\subsection{Pull Up Method}
\begin{frame}
  \frametitle{Pull Up Method}
  \begin{itemize}
	\item<+-> When you have identical methods on subclasses...
	\item<+-> ..whenever there is duplication, you face the risk that an alteration to one will not be made to the other
	\item<+-> When there are similar methods.. start from generlize them (i.e. extract same behavior and parameterize)
  \end{itemize}
\end{frame}

\subsection{Substitute Algorithm}
\begin{frame}
  \frametitle{Substitute Algorithm}
  \begin{itemize}
	\item<+-> When you want to replace an algorithm with one that is clearer..
	\item<+-> ...can occurs as you learn more about a problem and realize that there's an easier way to do it.
	\item<+-> ... It also happens if you start using a library that supplies features that duplicate your code
  \end{itemize}
\end{frame}

\subsection{Extract Class}
\begin{frame}
  \frametitle{Extract Class}
  \begin{itemize}
	\item<+-> A class can grow during his lifecycle.. a big class has too many responsibilities..
	\item<+-> ...split cohesive parts in a new one
  \end{itemize}
\end{frame}

\subsection{Replace Temp With Query}
\begin{frame}
  \frametitle{Replace Temp With Query}
  \begin{itemize}
	\item<+-> When you have a temp var holding a result of an expression
	\item<+-> ...extracting in a method can be re used..
	\item<+-> Temps encourage longer methods and are an obstacle to \textit{Extract Method}
  \end{itemize}
\end{frame}

\subsection{Introduce Parameter Object}
\begin{frame}
  \frametitle{Introduce Parameter Object}
  \begin{itemize}
	\item<+-> When you have a group of parameters that naturally go together
	\item<+-> ...replace them with an object
	\item<+-> At first you have a \textit{Data Clump}, but this is ready to receive behavior...
  \end{itemize}
\end{frame}

\subsection{Preserve Whole Object}
\begin{frame}
  \frametitle{Preserve Whole Object}
  \begin{itemize}
	\item<+-> When you are getting several values from an object and passing these values as parameters in a method call...
	\item<+-> ...send the whole object instead
	\item<+-> ..a need for new data in the mehod implies a change to the signature..
	\item<+-> Don't use you if you want to keep low coupling between caller and callee
  \end{itemize}
\end{frame}

\subsection{Replace Method With Method Object}
\begin{frame}
  \frametitle{Replace Method With Method Object}
  \begin{itemize}
	\item<+-> When you have a long method that uses local variables in such a way that you cannot apply \textit{Extract Method}..
	\item<+-> ...turn the method into its own object so that all the local variables become fields on that object
   \end{itemize}
\end{frame}

\subsection{Decompose Conditional}
\begin{frame}
  \frametitle{Decompose Conditionals}
  \begin{itemize}
	\item<+-> When you have a complicated conditional (if-then-else) statement...
	\item<+-> ...Extract methods from the condition, then part, and else parts
	\item<+-> Conditional logic is the most common area of complexity.. 
   \end{itemize}
\end{frame}

\end{document}